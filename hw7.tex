\documentclass{article}
\usepackage{amsmath}
\usepackage{enumerate}
\usepackage{fancyhdr} % Required for custom headers
\usepackage{lastpage} % Required to determine the last page for the footer
\usepackage{extramarks} % Required for headers and footers
\usepackage[usenames,dvipsnames]{color} % Required for custom colors
\usepackage{graphicx} % Required to insert images
\usepackage[tight,footnotesize]{subfigure} % Required for subfig
\usepackage{caption} % Required for subfig
\usepackage{hyperref} % Required for url
\usepackage{listings} % Required for insertion of code
\usepackage{courier} % Required for the courier font
\usepackage{lipsum} % Used for inserting dummy 'Lorem ipsum' text into the template
\usepackage[table,xcdraw]{xcolor}
\topmargin=-0.45in
\evensidemargin=0in
\oddsidemargin=0in
\textwidth=6.5in
\textheight=9.0in
\headsep=0.25in
\linespread{1.1} % Line spacing
\pagestyle{fancy}
\lhead{\hmwkAuthorName} % Top left header
% \chead{\hmwkClass\ (\hmwkClassInstructor\ \hmwkClassTime): \hmwkTitle} % Top center head
\chead{\hmwkClass\ : \hmwkTitle} % Top center head
\rhead{\firstxmark} % Top right header
\lfoot{\lastxmark} % Bottom left footer
\cfoot{} % Bottom center footer
\rfoot{Page\ \thepage\ of\ \protect\pageref{LastPage}} % Bottom right footer
\renewcommand\headrulewidth{0.4pt} % Size of the header rule
\renewcommand\footrulewidth{0.4pt} % Size of the footer rule
\setlength\parindent{0pt} % Removes all indentation from paragraphs

% Define floor and ceiling
\def\lc{\left\lceil}   
\def\rc{\right\rceil}
\def\lf{\left\lfloor}   
\def\rf{\right\rfloor}

% Set your language 
%\lstset{language=Java}
\definecolor{codegreen}{rgb}{0,0.6,0}
\definecolor{codegray}{rgb}{0.5,0.5,0.5}
\definecolor{codepurple}{rgb}{0.58,0,0.82}
\definecolor{backcolour}{rgb}{0.95,0.95,0.92}
 
\lstdefinestyle{mystyle}{
    backgroundcolor=\color{backcolour},   
    commentstyle=\color{codegreen},
    keywordstyle=\color{magenta},
    numberstyle=\tiny\color{codegray},
    stringstyle=\color{codepurple},
    basicstyle=\footnotesize,
    breakatwhitespace=false,         
    breaklines=true,                 
    captionpos=b,                    
    keepspaces=true,                 
    numbers=left,                    
    numbersep=8pt,                  
    showspaces=false,                
    showstringspaces=false,
    showtabs=false,                  
    tabsize=2
}
\lstset{style=mystyle}

% Header and footer for when a page split occurs within a problem environment
\newcommand{\enterProblemHeader}[1]{
\nobreak\extramarks{#1}{#1 continued on next page\ldots}\nobreak
\nobreak\extramarks{#1 (continued)}{#1 continued on next page\ldots}\nobreak
}

% Header and footer for when a page split occurs between problem environments
\newcommand{\exitProblemHeader}[1]{
\nobreak\extramarks{#1 (continued)}{#1 continued on next page\ldots}\nobreak
\nobreak\extramarks{#1}{}\nobreak
}

\setcounter{secnumdepth}{0} % Removes default section numbers
\newcounter{homeworkProblemCounter} % Creates a counter to keep track of the number of problems

\newcommand{\homeworkProblemName}{}
\newenvironment{homeworkProblem}[1][Problem \arabic{homeworkProblemCounter}]{ % Makes a new environment called homeworkProblem which takes 1 argument (custom name) but the default is "Problem #"
\stepcounter{homeworkProblemCounter} % Increase counter for number of problems
\renewcommand{\homeworkProblemName}{#1} % Assign \homeworkProblemName the name of the problem
\section{\homeworkProblemName} % Make a section in the document with the custom problem count
\enterProblemHeader{\homeworkProblemName} % Header and footer within the environment
}{
\exitProblemHeader{\homeworkProblemName} % Header and footer after the environment
}

\newcommand{\problemAnswer}[1]{ % Defines the problem answer command with the content as the only argument
\noindent\framebox[\columnwidth][c]{\begin{minipage}{0.98\columnwidth}#1\end{minipage}} % Makes the box around the problem answer and puts the content inside
}

\newcommand{\homeworkSectionName}{}
\newenvironment{homeworkSection}[1]{ % New environment for sections within homework problems, takes 1 argument - the name of the section
\renewcommand{\homeworkSectionName}{#1} % Assign \homeworkSectionName to the name of the section from the environment argument
\subsection{\homeworkSectionName} % Make a subsection with the custom name of the subsection
\enterProblemHeader{\homeworkProblemName\ [\homeworkSectionName]} % Header and footer within the environment
}{
\enterProblemHeader{\homeworkProblemName} % Header and footer after the environment
}

\newlength{\tabcont}

\newcommand{\tab}[1]{%
\settowidth{\tabcont}{#1}%
\ifthenelse{\lengthtest{\tabcont < .25\linewidth}}%
{\makebox[.25\linewidth][l]{#1}\ignorespaces}%
{\makebox[.5\linewidth][l]{\color{red} #1}\ignorespaces}%
}%
%----------------------------------------------------------------------------------------
%	NAME AND CLASS SECTION
%----------------------------------------------------------------------------------------

\newcommand{\hmwkTitle}{Homework\ \#7} % Assignment title
\newcommand{\hmwkDueDate}{Monday,\ January\ 1,\ 2012} % Due date
\newcommand{\hmwkClass}{Fundamental Algorithms} % Course/class
\newcommand{\hmwkClassTime}{} % Class/lecture time
\newcommand{\hmwkClassInstructor}{Prof. Joel Spencer} % Teacher/lecturer
\newcommand{\hmwkAuthorName}{Songxiao Zhang, N10224459, {\tt 72}} % Your name

%----------------------------------------------------------------------------------------
%	TITLE PAGE
%----------------------------------------------------------------------------------------

\title{
\textmd{\textbf{\hmwkClass:\ \hmwkTitle}}\\
}
\author{\textbf{\hmwkAuthorName}}

\begin{document}

\maketitle

%----------------------------------------------------------------------------------------
%	PROBLEM 1
%----------------------------------------------------------------------------------------
\begin{homeworkProblem}
First, fill in the grid with $X = [1, 0, 0, 1, 0, 1, 0, 1]$ and $Y = [0, 1, 0, 1, 1, 0, 1, 1, 0]$. 

% http://www.tablesgenerator.com/

\begin{table}[h]
\begin{tabular}{llllllllllll}
  &                        & 0                      & 1                      & 2                      & 3                      & 4                      & 5                      & 6                      & 7                      & 8                      & 9                      \\
  &                        & $y_i$                      & 0                      & 1                      & 0                      & 1                      & 1                      & 0                      & 1                      & 1                      & 0                      \\ \cline{3-12} 
0 & \multicolumn{1}{l|}{$x_i$} & \multicolumn{1}{l|}{0} & \multicolumn{1}{l|}{0} & \multicolumn{1}{l|}{0} & \multicolumn{1}{l|}{0} & \multicolumn{1}{l|}{0} & \multicolumn{1}{l|}{0} & \multicolumn{1}{l|}{0} & \multicolumn{1}{l|}{0} & \multicolumn{1}{l|}{0} & \multicolumn{1}{l|}{0} \\ \cline{3-12} 
1 & \multicolumn{1}{l|}{1} & \multicolumn{1}{l|}{0} & \multicolumn{1}{l|}{0} & \multicolumn{1}{l|}{1} & \multicolumn{1}{l|}{1} & \multicolumn{1}{l|}{1} & \multicolumn{1}{l|}{1} & \multicolumn{1}{l|}{1} & \multicolumn{1}{l|}{1} & \multicolumn{1}{l|}{1} & \multicolumn{1}{l|}{1} \\ \cline{3-12} 
2 & \multicolumn{1}{l|}{0} & \multicolumn{1}{l|}{0} & \multicolumn{1}{l|}{1} & \multicolumn{1}{l|}{1} & \multicolumn{1}{l|}{2} & \multicolumn{1}{l|}{2} & \multicolumn{1}{l|}{2} & \multicolumn{1}{l|}{2} & \multicolumn{1}{l|}{2} & \multicolumn{1}{l|}{2} & \multicolumn{1}{l|}{2} \\ \cline{3-12} 
3 & \multicolumn{1}{l|}{0} & \multicolumn{1}{l|}{0} & \multicolumn{1}{l|}{1} & \multicolumn{1}{l|}{1} & \multicolumn{1}{l|}{2} & \multicolumn{1}{l|}{2} & \multicolumn{1}{l|}{2} & \multicolumn{1}{l|}{3} & \multicolumn{1}{l|}{3} & \multicolumn{1}{l|}{3} & \multicolumn{1}{l|}{3} \\ \cline{3-12} 
4 & \multicolumn{1}{l|}{1} & \multicolumn{1}{l|}{0} & \multicolumn{1}{l|}{1} & \multicolumn{1}{l|}{2} & \multicolumn{1}{l|}{2} & \multicolumn{1}{l|}{3} & \multicolumn{1}{l|}{3} & \multicolumn{1}{l|}{3} & \multicolumn{1}{l|}{4} & \multicolumn{1}{l|}{4} & \multicolumn{1}{l|}{4} \\ \cline{3-12} 
5 & \multicolumn{1}{l|}{0} & \multicolumn{1}{l|}{0} & \multicolumn{1}{l|}{1} & \multicolumn{1}{l|}{2} & \multicolumn{1}{l|}{3} & \multicolumn{1}{l|}{3} & \multicolumn{1}{l|}{3} & \multicolumn{1}{l|}{4} & \multicolumn{1}{l|}{4} & \multicolumn{1}{l|}{4} & \multicolumn{1}{l|}{5} \\ \cline{3-12} 
6 & \multicolumn{1}{l|}{1} & \multicolumn{1}{l|}{0} & \multicolumn{1}{l|}{1} & \multicolumn{1}{l|}{2} & \multicolumn{1}{l|}{3} & \multicolumn{1}{l|}{4} & \multicolumn{1}{l|}{4} & \multicolumn{1}{l|}{4} & \multicolumn{1}{l|}{5} & \multicolumn{1}{l|}{5} & \multicolumn{1}{l|}{5} \\ \cline{3-12} 
7 & \multicolumn{1}{l|}{0} & \multicolumn{1}{l|}{0} & \multicolumn{1}{l|}{1} & \multicolumn{1}{l|}{2} & \multicolumn{1}{l|}{3} & \multicolumn{1}{l|}{4} & \multicolumn{1}{l|}{4} & \multicolumn{1}{l|}{5} & \multicolumn{1}{l|}{5} & \multicolumn{1}{l|}{5} & \multicolumn{1}{l|}{6} \\ \cline{3-12} 
8 & \multicolumn{1}{l|}{1} & \multicolumn{1}{l|}{0} & \multicolumn{1}{l|}{1} & \multicolumn{1}{l|}{2} & \multicolumn{1}{l|}{3} & \multicolumn{1}{l|}{4} & \multicolumn{1}{l|}{5} & \multicolumn{1}{l|}{5} & \multicolumn{1}{l|}{6} & \multicolumn{1}{l|}{6} & \multicolumn{1}{l|}{6} \\ \cline{3-12} 
\end{tabular}
\end{table}

Then, fill the table with appropriate arrows (took me hours doing this). 

\begin{table}[h]
\begin{tabular}{llllllllllll}
 &  & 0 & 1 & 2 & 3 & 4 & 5 & 6 & 7 & 8 & 9 \\
 &  & $y_i$ & 0 & 1 & 0 & 1 & 1 & 0 & 1 & 1 & 0 \\ \cline{3-12} 
0 & \multicolumn{1}{l|}{$x_i$} & \multicolumn{1}{l|}{0} & \multicolumn{1}{l|}{0} & \multicolumn{1}{l|}{0} & \multicolumn{1}{l|}{0} & \multicolumn{1}{l|}{0} & \multicolumn{1}{l|}{0} & \multicolumn{1}{l|}{0} & \multicolumn{1}{l|}{0} & \multicolumn{1}{l|}{0} & \multicolumn{1}{l|}{0} \\ \cline{3-12} 
1 & \multicolumn{1}{l|}{1} & \multicolumn{1}{l|}{0} & \multicolumn{1}{l|}{\begin{tabular}[c]{@{}l@{}}$\uparrow$\\ 0\end{tabular}} & \multicolumn{1}{l|}{\begin{tabular}[c]{@{}l@{}}$\nwarrow$\\ 1\end{tabular}} & \multicolumn{1}{l|}{\begin{tabular}[c]{@{}l@{}}$\leftarrow$\\ 1\end{tabular}} & \multicolumn{1}{l|}{\begin{tabular}[c]{@{}l@{}}$\nwarrow$\\ 1\end{tabular}} & \multicolumn{1}{l|}{\begin{tabular}[c]{@{}l@{}}$\nwarrow$\\ 1\end{tabular}} & \multicolumn{1}{l|}{\begin{tabular}[c]{@{}l@{}}$\leftarrow$\\ 1\end{tabular}} & \multicolumn{1}{l|}{\begin{tabular}[c]{@{}l@{}}$\nwarrow$\\ 1\end{tabular}} & \multicolumn{1}{l|}{\begin{tabular}[c]{@{}l@{}}$\nwarrow$\\ 1\end{tabular}} & \multicolumn{1}{l|}{\begin{tabular}[c]{@{}l@{}}$\leftarrow$\\ 1\end{tabular}} \\ \cline{3-12} 
2 & \multicolumn{1}{l|}{0} & \multicolumn{1}{l|}{0} & \multicolumn{1}{l|}{\begin{tabular}[c]{@{}l@{}}$\nwarrow$\\ 1\end{tabular}} & \multicolumn{1}{l|}{\begin{tabular}[c]{@{}l@{}}$\uparrow$\\ 1\end{tabular}} & \multicolumn{1}{l|}{\begin{tabular}[c]{@{}l@{}}$\nwarrow$\\ 2\end{tabular}} & \multicolumn{1}{l|}{\begin{tabular}[c]{@{}l@{}}$\leftarrow$\\ 2\end{tabular}} & \multicolumn{1}{l|}{\begin{tabular}[c]{@{}l@{}}$\leftarrow$\\ 2\end{tabular}} & \multicolumn{1}{l|}{\begin{tabular}[c]{@{}l@{}}$\nwarrow$\\ 2\end{tabular}} & \multicolumn{1}{l|}{\begin{tabular}[c]{@{}l@{}}$\leftarrow$\\ 2\end{tabular}} & \multicolumn{1}{l|}{\begin{tabular}[c]{@{}l@{}}$\leftarrow$\\ 2\end{tabular}} & \multicolumn{1}{l|}{\begin{tabular}[c]{@{}l@{}}$\nearrow$\\ 2\end{tabular}} \\ \cline{3-12} 
3 & \multicolumn{1}{l|}{0} & \multicolumn{1}{l|}{0} & \multicolumn{1}{l|}{\begin{tabular}[c]{@{}l@{}}$\nwarrow$\\ 1\end{tabular}} & \multicolumn{1}{l|}{\begin{tabular}[c]{@{}l@{}}$\nwarrow$\\ 1\end{tabular}} & \multicolumn{1}{l|}{\begin{tabular}[c]{@{}l@{}}$\nwarrow$\\ 2\end{tabular}} & \multicolumn{1}{l|}{\begin{tabular}[c]{@{}l@{}}$\uparrow$\\ 2\end{tabular}} & \multicolumn{1}{l|}{\begin{tabular}[c]{@{}l@{}}$\uparrow$\\ 2\end{tabular}} & \multicolumn{1}{l|}{\begin{tabular}[c]{@{}l@{}}$\nwarrow$\\ 3\end{tabular}} & \multicolumn{1}{l|}{\begin{tabular}[c]{@{}l@{}}$\leftarrow$\\ 3\end{tabular}} & \multicolumn{1}{l|}{\begin{tabular}[c]{@{}l@{}}$\leftarrow$\\ 3\end{tabular}} & \multicolumn{1}{l|}{\begin{tabular}[c]{@{}l@{}}$\nwarrow$\\ 3\end{tabular}} \\ \cline{3-12} 
4 & \multicolumn{1}{l|}{1} & \multicolumn{1}{l|}{0} & \multicolumn{1}{l|}{\begin{tabular}[c]{@{}l@{}}$\uparrow$\\ 1\end{tabular}} & \multicolumn{1}{l|}{\begin{tabular}[c]{@{}l@{}}$\nwarrow$\\ 2\end{tabular}} & \multicolumn{1}{l|}{\begin{tabular}[c]{@{}l@{}}$\uparrow$\\ 2\end{tabular}} & \multicolumn{1}{l|}{\begin{tabular}[c]{@{}l@{}}$\nwarrow$\\ 3\end{tabular}} & \multicolumn{1}{l|}{\begin{tabular}[c]{@{}l@{}}$\nwarrow$\\ 3\end{tabular}} & \multicolumn{1}{l|}{\begin{tabular}[c]{@{}l@{}}$\uparrow$\\ 3\end{tabular}} & \multicolumn{1}{l|}{\begin{tabular}[c]{@{}l@{}}$\nwarrow$\\ 4\end{tabular}} & \multicolumn{1}{l|}{\begin{tabular}[c]{@{}l@{}}$\nwarrow$\\ 4\end{tabular}} & \multicolumn{1}{l|}{\begin{tabular}[c]{@{}l@{}}$\leftarrow$\\ 4\end{tabular}} \\ \cline{3-12} 
5 & \multicolumn{1}{l|}{0} & \multicolumn{1}{l|}{0} & \multicolumn{1}{l|}{\begin{tabular}[c]{@{}l@{}}$\nwarrow$\\ 1\end{tabular}} & \multicolumn{1}{l|}{\begin{tabular}[c]{@{}l@{}}$\uparrow$\\ 2\end{tabular}} & \multicolumn{1}{l|}{\begin{tabular}[c]{@{}l@{}}$\nwarrow$\\ 3\end{tabular}} & \multicolumn{1}{l|}{\begin{tabular}[c]{@{}l@{}}$\uparrow$\\ 3\end{tabular}} & \multicolumn{1}{l|}{\begin{tabular}[c]{@{}l@{}}$\uparrow$\\ 3\end{tabular}} & \multicolumn{1}{l|}{\begin{tabular}[c]{@{}l@{}}$\nwarrow$\\ 4\end{tabular}} & \multicolumn{1}{l|}{\begin{tabular}[c]{@{}l@{}}$\uparrow$\\ 4\end{tabular}} & \multicolumn{1}{l|}{\begin{tabular}[c]{@{}l@{}}$\uparrow$\\ 4\end{tabular}} & \multicolumn{1}{l|}{\begin{tabular}[c]{@{}l@{}}$\nwarrow$\\ 5\end{tabular}} \\ \cline{3-12} 
6 & \multicolumn{1}{l|}{1} & \multicolumn{1}{l|}{0} & \multicolumn{1}{l|}{\begin{tabular}[c]{@{}l@{}}$\uparrow$\\ 1\end{tabular}} & \multicolumn{1}{l|}{\begin{tabular}[c]{@{}l@{}}$\nwarrow$\\ 2\end{tabular}} & \multicolumn{1}{l|}{\begin{tabular}[c]{@{}l@{}}$\uparrow$\\ 3\end{tabular}} & \multicolumn{1}{l|}{\begin{tabular}[c]{@{}l@{}}$\nwarrow$\\ 4\end{tabular}} & \multicolumn{1}{l|}{\begin{tabular}[c]{@{}l@{}}$\nwarrow$\\ 4\end{tabular}} & \multicolumn{1}{l|}{\begin{tabular}[c]{@{}l@{}}$\uparrow$\\ 4\end{tabular}} & \multicolumn{1}{l|}{\begin{tabular}[c]{@{}l@{}}$\nwarrow$\\ 5\end{tabular}} & \multicolumn{1}{l|}{\begin{tabular}[c]{@{}l@{}}$\nwarrow$\\ 5\end{tabular}} & \multicolumn{1}{l|}{\begin{tabular}[c]{@{}l@{}}$\uparrow$\\ 5\end{tabular}} \\ \cline{3-12} 
7 & \multicolumn{1}{l|}{0} & \multicolumn{1}{l|}{0} & \multicolumn{1}{l|}{\begin{tabular}[c]{@{}l@{}}$\nwarrow$\\ 1\end{tabular}} & \multicolumn{1}{l|}{\begin{tabular}[c]{@{}l@{}}$\uparrow$\\ 2\end{tabular}} & \multicolumn{1}{l|}{\begin{tabular}[c]{@{}l@{}}$\nwarrow$\\ 3\end{tabular}} & \multicolumn{1}{l|}{\begin{tabular}[c]{@{}l@{}}$\uparrow$\\ 4\end{tabular}} & \multicolumn{1}{l|}{\begin{tabular}[c]{@{}l@{}}$\uparrow$\\ 4\end{tabular}} & \multicolumn{1}{l|}{\begin{tabular}[c]{@{}l@{}}$\nwarrow$\\ 5\end{tabular}} & \multicolumn{1}{l|}{\begin{tabular}[c]{@{}l@{}}$\uparrow$\\ 5\end{tabular}} & \multicolumn{1}{l|}{\begin{tabular}[c]{@{}l@{}}$\uparrow$\\ 5\end{tabular}} & \multicolumn{1}{l|}{\begin{tabular}[c]{@{}l@{}}$\nwarrow$\\ 6\end{tabular}} \\ \cline{3-12} 
8 & \multicolumn{1}{l|}{1} & \multicolumn{1}{l|}{0} & \multicolumn{1}{l|}{\begin{tabular}[c]{@{}l@{}}$\uparrow$\\ 1\end{tabular}} & \multicolumn{1}{l|}{\begin{tabular}[c]{@{}l@{}}$\nwarrow$\\ 2\end{tabular}} & \multicolumn{1}{l|}{\begin{tabular}[c]{@{}l@{}}$\uparrow$\\ 3\end{tabular}} & \multicolumn{1}{l|}{\begin{tabular}[c]{@{}l@{}}$\nwarrow$\\ 4\end{tabular}} & \multicolumn{1}{l|}{\begin{tabular}[c]{@{}l@{}}$\nwarrow$\\ 5\end{tabular}} & \multicolumn{1}{l|}{\begin{tabular}[c]{@{}l@{}}$\uparrow$\\ 5\end{tabular}} & \multicolumn{1}{l|}{\begin{tabular}[c]{@{}l@{}}$\nwarrow$\\ 6\end{tabular}} & \multicolumn{1}{l|}{\begin{tabular}[c]{@{}l@{}}$\nwarrow$\\ 6\end{tabular}} & \multicolumn{1}{l|}{\begin{tabular}[c]{@{}l@{}}$\uparrow$\\ 6\end{tabular}} \\ \cline{3-12} 
\end{tabular}
\end{table}

% % Please add the following required packages to your document preamble:
% % \usepackage[table,xcdraw]{xcolor}
% % If you use beamer only pass "xcolor=table" option, i.e. \documentclass[xcolor=table]{beamer}
% \begin{table}[h]
% \begin{tabular}{llllllllllll}
%  &  & 0 & 1 & 2 & 3 & 4 & 5 & 6 & 7 & 8 & 9 \\
%  &  & $y_i$ & 0 & 1 & 0 & 1 & 1 & 0 & 1 & 1 & 0 \\ \cline{3-12} 
% 0 & \multicolumn{1}{l|}{$x_i$} & \multicolumn{1}{l|}{0} & \multicolumn{1}{l|}{\cellcolor[HTML]{CBCEFB}\begin{tabular}[c]{@{}l@{}}$\nwarrow$\\ 0\end{tabular}} & \multicolumn{1}{l|}{0} & \multicolumn{1}{l|}{0} & \multicolumn{1}{l|}{0} & \multicolumn{1}{l|}{0} & \multicolumn{1}{l|}{0} & \multicolumn{1}{l|}{0} & \multicolumn{1}{l|}{0} & \multicolumn{1}{l|}{0} \\ \cline{3-12} 
% 1 & \multicolumn{1}{l|}{1} & \multicolumn{1}{l|}{0} & \multicolumn{1}{l|}{0} & \multicolumn{1}{l|}{\cellcolor[HTML]{CBCEFB}\begin{tabular}[c]{@{}l@{}}$\nwarrow$\\ 1\end{tabular}} & \multicolumn{1}{l|}{1} & \multicolumn{1}{l|}{1} & \multicolumn{1}{l|}{1} & \multicolumn{1}{l|}{1} & \multicolumn{1}{l|}{1} & \multicolumn{1}{l|}{1} & \multicolumn{1}{l|}{1} \\ \cline{3-12} 
% 2 & \multicolumn{1}{l|}{0} & \multicolumn{1}{l|}{0} & \multicolumn{1}{l|}{1} & \multicolumn{1}{l|}{1} & \multicolumn{1}{l|}{\cellcolor[HTML]{CBCEFB}\begin{tabular}[c]{@{}l@{}}$\nwarrow$\\ 2\end{tabular}} & \multicolumn{1}{l|}{\cellcolor[HTML]{CBCEFB}\begin{tabular}[c]{@{}l@{}}$\leftarrow$\\ 2\end{tabular}} & \multicolumn{1}{l|}{\cellcolor[HTML]{CBCEFB}\begin{tabular}[c]{@{}l@{}}$\leftarrow$\\ 2\end{tabular}} & \multicolumn{1}{l|}{2} & \multicolumn{1}{l|}{2} & \multicolumn{1}{l|}{2} & \multicolumn{1}{l|}{2} \\ \cline{3-12} 
% 3 & \multicolumn{1}{l|}{0} & \multicolumn{1}{l|}{0} & \multicolumn{1}{l|}{1} & \multicolumn{1}{l|}{1} & \multicolumn{1}{l|}{2} & \multicolumn{1}{l|}{2} & \multicolumn{1}{l|}{2} & \multicolumn{1}{l|}{\cellcolor[HTML]{CBCEFB}\begin{tabular}[c]{@{}l@{}}$\nwarrow$\\ 3\end{tabular}} & \multicolumn{1}{l|}{3} & \multicolumn{1}{l|}{3} & \multicolumn{1}{l|}{3} \\ \cline{3-12} 
% 4 & \multicolumn{1}{l|}{1} & \multicolumn{1}{l|}{0} & \multicolumn{1}{l|}{1} & \multicolumn{1}{l|}{2} & \multicolumn{1}{l|}{2} & \multicolumn{1}{l|}{3} & \multicolumn{1}{l|}{3} & \multicolumn{1}{l|}{3} & \multicolumn{1}{l|}{\cellcolor[HTML]{CBCEFB}\begin{tabular}[c]{@{}l@{}}$\nwarrow$\\ 4\end{tabular}} & \multicolumn{1}{l|}{4} & \multicolumn{1}{l|}{4} \\ \cline{3-12} 
% 5 & \multicolumn{1}{l|}{0} & \multicolumn{1}{l|}{0} & \multicolumn{1}{l|}{1} & \multicolumn{1}{l|}{2} & \multicolumn{1}{l|}{3} & \multicolumn{1}{l|}{3} & \multicolumn{1}{l|}{3} & \multicolumn{1}{l|}{4} & \multicolumn{1}{l|}{\cellcolor[HTML]{CBCEFB}\begin{tabular}[c]{@{}l@{}}$\uparrow$\\ 4\end{tabular}} & \multicolumn{1}{l|}{4} & \multicolumn{1}{l|}{5} \\ \cline{3-12} 
% 6 & \multicolumn{1}{l|}{1} & \multicolumn{1}{l|}{0} & \multicolumn{1}{l|}{1} & \multicolumn{1}{l|}{2} & \multicolumn{1}{l|}{3} & \multicolumn{1}{l|}{4} & \multicolumn{1}{l|}{4} & \multicolumn{1}{l|}{4} & \multicolumn{1}{l|}{5} & \multicolumn{1}{l|}{\cellcolor[HTML]{CBCEFB}\begin{tabular}[c]{@{}l@{}}$\nwarrow$\\ 5\end{tabular}} & \multicolumn{1}{l|}{\begin{tabular}[c]{@{}l@{}}$\leftarrow$\\ 5\end{tabular}} \\ \cline{3-12} 
% 7 & \multicolumn{1}{l|}{0} & \multicolumn{1}{l|}{0} & \multicolumn{1}{l|}{1} & \multicolumn{1}{l|}{2} & \multicolumn{1}{l|}{3} & \multicolumn{1}{l|}{4} & \multicolumn{1}{l|}{4} & \multicolumn{1}{l|}{5} & \multicolumn{1}{l|}{5} & \multicolumn{1}{l|}{5} & \multicolumn{1}{l|}{\cellcolor[HTML]{CBCEFB}\begin{tabular}[c]{@{}l@{}}$\nwarrow$\\ 6\end{tabular}} \\ \cline{3-12} 
% 8 & \multicolumn{1}{l|}{1} & \multicolumn{1}{l|}{0} & \multicolumn{1}{l|}{1} & \multicolumn{1}{l|}{2} & \multicolumn{1}{l|}{3} & \multicolumn{1}{l|}{4} & \multicolumn{1}{l|}{5} & \multicolumn{1}{l|}{5} & \multicolumn{1}{l|}{6} & \multicolumn{1}{l|}{6} & \multicolumn{1}{l|}{\cellcolor[HTML]{CBCEFB}\begin{tabular}[c]{@{}l@{}}$\uparrow$\\ 6\end{tabular}} \\ \cline{3-12} 
% \end{tabular}
% \end{table}


In the end, draw a path from the right bottom corner until hit the wall of $x_i$ or $y_i$. Collect all $\nwarrow$ to form the path $[1, 0, 0, 1, 1, 0]$

% Please add the following required packages to your document preamble:
% \usepackage[table,xcdraw]{xcolor}
% If you use beamer only pass "xcolor=table" option, i.e. \documentclass[xcolor=table]{beamer}
\begin{table}[h]
\begin{tabular}{llllllllllll}
 &  & 0 & 1 & 2 & 3 & 4 & 5 & 6 & 7 & 8 & 9 \\
 &  & $y_i$ & 0 & 1 & 0 & 1 & 1 & 0 & 1 & 1 & 0 \\ \cline{3-12} 
0 & \multicolumn{1}{l|}{$x_i$} & \multicolumn{1}{l|}{0} & \multicolumn{1}{l|}{\cellcolor[HTML]{CBCEFB}0} & \multicolumn{1}{l|}{0} & \multicolumn{1}{l|}{0} & \multicolumn{1}{l|}{0} & \multicolumn{1}{l|}{0} & \multicolumn{1}{l|}{0} & \multicolumn{1}{l|}{0} & \multicolumn{1}{l|}{0} & \multicolumn{1}{l|}{0} \\ \cline{3-12} 
1 & \multicolumn{1}{l|}{1} & \multicolumn{1}{l|}{0} & \multicolumn{1}{l|}{\begin{tabular}[c]{@{}l@{}}$\uparrow$\\ 0\end{tabular}} & \multicolumn{1}{l|}{\cellcolor[HTML]{CBCEFB}\begin{tabular}[c]{@{}l@{}}$\nwarrow$\\ 1\end{tabular}} & \multicolumn{1}{l|}{\begin{tabular}[c]{@{}l@{}}$\leftarrow$\\ 1\end{tabular}} & \multicolumn{1}{l|}{\begin{tabular}[c]{@{}l@{}}$\nwarrow$\\ 1\end{tabular}} & \multicolumn{1}{l|}{\begin{tabular}[c]{@{}l@{}}$\nwarrow$\\ 1\end{tabular}} & \multicolumn{1}{l|}{\begin{tabular}[c]{@{}l@{}}$\leftarrow$\\ 1\end{tabular}} & \multicolumn{1}{l|}{\begin{tabular}[c]{@{}l@{}}$\nwarrow$\\ 1\end{tabular}} & \multicolumn{1}{l|}{\begin{tabular}[c]{@{}l@{}}$\nwarrow$\\ 1\end{tabular}} & \multicolumn{1}{l|}{\begin{tabular}[c]{@{}l@{}}$\leftarrow$\\ 1\end{tabular}} \\ \cline{3-12} 
2 & \multicolumn{1}{l|}{0} & \multicolumn{1}{l|}{0} & \multicolumn{1}{l|}{\begin{tabular}[c]{@{}l@{}}$\nwarrow$\\ 1\end{tabular}} & \multicolumn{1}{l|}{\begin{tabular}[c]{@{}l@{}}$\uparrow$\\ 1\end{tabular}} & \multicolumn{1}{l|}{\cellcolor[HTML]{CBCEFB}\begin{tabular}[c]{@{}l@{}}$\nwarrow$\\ 2\end{tabular}} & \multicolumn{1}{l|}{\cellcolor[HTML]{CBCEFB}\begin{tabular}[c]{@{}l@{}}$\leftarrow$\\ 2\end{tabular}} & \multicolumn{1}{l|}{\cellcolor[HTML]{CBCEFB}\begin{tabular}[c]{@{}l@{}}$\leftarrow$\\ 2\end{tabular}} & \multicolumn{1}{l|}{\begin{tabular}[c]{@{}l@{}}$\nwarrow$\\ 2\end{tabular}} & \multicolumn{1}{l|}{\begin{tabular}[c]{@{}l@{}}$\leftarrow$\\ 2\end{tabular}} & \multicolumn{1}{l|}{\begin{tabular}[c]{@{}l@{}}$\leftarrow$\\ 2\end{tabular}} & \multicolumn{1}{l|}{\begin{tabular}[c]{@{}l@{}}$\nearrow$\\ 2\end{tabular}} \\ \cline{3-12} 
3 & \multicolumn{1}{l|}{0} & \multicolumn{1}{l|}{0} & \multicolumn{1}{l|}{\begin{tabular}[c]{@{}l@{}}$\nwarrow$\\ 1\end{tabular}} & \multicolumn{1}{l|}{\begin{tabular}[c]{@{}l@{}}$\nwarrow$\\ 1\end{tabular}} & \multicolumn{1}{l|}{\begin{tabular}[c]{@{}l@{}}$\nwarrow$\\ 2\end{tabular}} & \multicolumn{1}{l|}{\begin{tabular}[c]{@{}l@{}}$\uparrow$\\ 2\end{tabular}} & \multicolumn{1}{l|}{\begin{tabular}[c]{@{}l@{}}$\uparrow$\\ 2\end{tabular}} & \multicolumn{1}{l|}{\cellcolor[HTML]{CBCEFB}\begin{tabular}[c]{@{}l@{}}$\nwarrow$\\ 3\end{tabular}} & \multicolumn{1}{l|}{\begin{tabular}[c]{@{}l@{}}$\leftarrow$\\ 3\end{tabular}} & \multicolumn{1}{l|}{\begin{tabular}[c]{@{}l@{}}$\leftarrow$\\ 3\end{tabular}} & \multicolumn{1}{l|}{\begin{tabular}[c]{@{}l@{}}$\nwarrow$\\ 3\end{tabular}} \\ \cline{3-12} 
4 & \multicolumn{1}{l|}{1} & \multicolumn{1}{l|}{0} & \multicolumn{1}{l|}{\begin{tabular}[c]{@{}l@{}}$\uparrow$\\ 1\end{tabular}} & \multicolumn{1}{l|}{\begin{tabular}[c]{@{}l@{}}$\nwarrow$\\ 2\end{tabular}} & \multicolumn{1}{l|}{\begin{tabular}[c]{@{}l@{}}$\uparrow$\\ 2\end{tabular}} & \multicolumn{1}{l|}{\begin{tabular}[c]{@{}l@{}}$\nwarrow$\\ 3\end{tabular}} & \multicolumn{1}{l|}{\begin{tabular}[c]{@{}l@{}}$\nwarrow$\\ 3\end{tabular}} & \multicolumn{1}{l|}{\begin{tabular}[c]{@{}l@{}}$\uparrow$\\ 3\end{tabular}} & \multicolumn{1}{l|}{\cellcolor[HTML]{CBCEFB}\begin{tabular}[c]{@{}l@{}}$\nwarrow$\\ 4\end{tabular}} & \multicolumn{1}{l|}{\begin{tabular}[c]{@{}l@{}}$\nwarrow$\\ 4\end{tabular}} & \multicolumn{1}{l|}{\begin{tabular}[c]{@{}l@{}}$\leftarrow$\\ 4\end{tabular}} \\ \cline{3-12} 
5 & \multicolumn{1}{l|}{0} & \multicolumn{1}{l|}{0} & \multicolumn{1}{l|}{\begin{tabular}[c]{@{}l@{}}$\nwarrow$\\ 1\end{tabular}} & \multicolumn{1}{l|}{\begin{tabular}[c]{@{}l@{}}$\uparrow$\\ 2\end{tabular}} & \multicolumn{1}{l|}{\begin{tabular}[c]{@{}l@{}}$\nwarrow$\\ 3\end{tabular}} & \multicolumn{1}{l|}{\begin{tabular}[c]{@{}l@{}}$\uparrow$\\ 3\end{tabular}} & \multicolumn{1}{l|}{\begin{tabular}[c]{@{}l@{}}$\uparrow$\\ 3\end{tabular}} & \multicolumn{1}{l|}{\begin{tabular}[c]{@{}l@{}}$\nwarrow$\\ 4\end{tabular}} & \multicolumn{1}{l|}{\cellcolor[HTML]{CBCEFB}\begin{tabular}[c]{@{}l@{}}$\uparrow$\\ 4\end{tabular}} & \multicolumn{1}{l|}{\begin{tabular}[c]{@{}l@{}}$\uparrow$\\ 4\end{tabular}} & \multicolumn{1}{l|}{\begin{tabular}[c]{@{}l@{}}$\nwarrow$\\ 5\end{tabular}} \\ \cline{3-12} 
6 & \multicolumn{1}{l|}{1} & \multicolumn{1}{l|}{0} & \multicolumn{1}{l|}{\begin{tabular}[c]{@{}l@{}}$\uparrow$\\ 1\end{tabular}} & \multicolumn{1}{l|}{\begin{tabular}[c]{@{}l@{}}$\nwarrow$\\ 2\end{tabular}} & \multicolumn{1}{l|}{\begin{tabular}[c]{@{}l@{}}$\uparrow$\\ 3\end{tabular}} & \multicolumn{1}{l|}{\begin{tabular}[c]{@{}l@{}}$\nwarrow$\\ 4\end{tabular}} & \multicolumn{1}{l|}{\begin{tabular}[c]{@{}l@{}}$\nwarrow$\\ 4\end{tabular}} & \multicolumn{1}{l|}{\begin{tabular}[c]{@{}l@{}}$\uparrow$\\ 4\end{tabular}} & \multicolumn{1}{l|}{\begin{tabular}[c]{@{}l@{}}$\nwarrow$\\ 5\end{tabular}} & \multicolumn{1}{l|}{\cellcolor[HTML]{CBCEFB}\begin{tabular}[c]{@{}l@{}}$\nwarrow$\\ 5\end{tabular}} & \multicolumn{1}{l|}{\begin{tabular}[c]{@{}l@{}}$\uparrow$\\ 5\end{tabular}} \\ \cline{3-12} 
7 & \multicolumn{1}{l|}{0} & \multicolumn{1}{l|}{0} & \multicolumn{1}{l|}{\begin{tabular}[c]{@{}l@{}}$\nwarrow$\\ 1\end{tabular}} & \multicolumn{1}{l|}{\begin{tabular}[c]{@{}l@{}}$\uparrow$\\ 2\end{tabular}} & \multicolumn{1}{l|}{\begin{tabular}[c]{@{}l@{}}$\nwarrow$\\ 3\end{tabular}} & \multicolumn{1}{l|}{\begin{tabular}[c]{@{}l@{}}$\uparrow$\\ 4\end{tabular}} & \multicolumn{1}{l|}{\begin{tabular}[c]{@{}l@{}}$\uparrow$\\ 4\end{tabular}} & \multicolumn{1}{l|}{\begin{tabular}[c]{@{}l@{}}$\nwarrow$\\ 5\end{tabular}} & \multicolumn{1}{l|}{\begin{tabular}[c]{@{}l@{}}$\uparrow$\\ 5\end{tabular}} & \multicolumn{1}{l|}{\begin{tabular}[c]{@{}l@{}}$\uparrow$\\ 5\end{tabular}} & \multicolumn{1}{l|}{\cellcolor[HTML]{CBCEFB}\begin{tabular}[c]{@{}l@{}}$\nwarrow$\\ 6\end{tabular}} \\ \cline{3-12} 
8 & \multicolumn{1}{l|}{1} & \multicolumn{1}{l|}{0} & \multicolumn{1}{l|}{\begin{tabular}[c]{@{}l@{}}$\uparrow$\\ 1\end{tabular}} & \multicolumn{1}{l|}{\begin{tabular}[c]{@{}l@{}}$\nwarrow$\\ 2\end{tabular}} & \multicolumn{1}{l|}{\begin{tabular}[c]{@{}l@{}}$\uparrow$\\ 3\end{tabular}} & \multicolumn{1}{l|}{\begin{tabular}[c]{@{}l@{}}$\nwarrow$\\ 4\end{tabular}} & \multicolumn{1}{l|}{\begin{tabular}[c]{@{}l@{}}$\nwarrow$\\ 5\end{tabular}} & \multicolumn{1}{l|}{\begin{tabular}[c]{@{}l@{}}$\uparrow$\\ 5\end{tabular}} & \multicolumn{1}{l|}{\begin{tabular}[c]{@{}l@{}}$\nwarrow$\\ 6\end{tabular}} & \multicolumn{1}{l|}{\begin{tabular}[c]{@{}l@{}}$\nwarrow$\\ 6\end{tabular}} & \multicolumn{1}{l|}{\cellcolor[HTML]{CBCEFB}\begin{tabular}[c]{@{}l@{}}$\uparrow$\\ 6\end{tabular}} \\ \cline{3-12} 
\end{tabular}
\end{table}

\end{homeworkProblem}

%----------------------------------------------------------------------------------------
%	PROBLEM 2
%----------------------------------------------------------------------------------------

\begin{homeworkProblem}
\begin{equation}
  P(n) = \frac{1}{2n-1}
  \begin{pmatrix}
    2n - 1 \\
    n - 1
  \end{pmatrix}
  = \sum{i=1}{n-1}P(i)P(n-i)
\end{equation}
$P(1) = 1, P(2) = 1, P(3) = 2, P(4) = 5 $
When $n = 4$, $P(5) = P(1)P(4) + P(2)P(3) + P(3)P(2) + P(4)P(1) = 2(1*5 + 1*2) = 14$. \\

i = 1, \\
$ A (B (C (D E))) $\\
$ A (B ((C D) E)) $\\
$ A ((B C) (D E)) $\\
$ A ((B (C D)) E) $\\
$ A (((B C) D) E) $\\

i = 2, \\
$ (A B) (C (D E)) $\\
$ (A B) ((C D) E) $\\

i = 3, \\
$ (A (B C)) (D E) $\\
$ ((A B) C) (D E) $\\

i = 4, \\
$ (A (B (C D))) E $\\
$ (A ((B C) D)) E $\\
$ ((A B) (C D)) E $\\
$ ((A (B C)) D) E $\\
$ (((A B) C) D) E $\\

\end{homeworkProblem}

%----------------------------------------------------------------------------------------
%	PROBLEM 3
%----------------------------------------------------------------------------------------

\begin{homeworkProblem}
\begin{enumerate}[(a)]
    % a
    \item The brute-force approach would be list all the sub-sequences end in 
          $x_1, x_2, ..., x_m$ and get the largest. But the running time would be exponential. 
          $O(2^n)$
          
          We can optimize it by a interesting finding that the longest sub-sequence size would 
          be increased by 1 if the next element would be larger than the maximum value of
          the longest sub-sequence on the left. 
          The logic behind it is $$INC[i] = 1 + max(INC[j])$$ where $1<j<i, A[i] > A[j]$. 
          We can construct a function inc(int[] A, int i, int max) for INC[i] to pass all the 
          data. Now the running time is reduced to $O(n^2)$ and code is listed below \\
          
\begin{lstlisting}[frame=single]
A = [x1, x2, ..., xm]
INC[i] = inc(A, i, 1)

inc(A, n)
    for i from 1 to (n - 1)
        lis[i] = 1
        
    for i form 1 to (n - 1)
        for j from 1 to (i - 1)
            if A[j] < A[i] && lis[i] < lis[j] + 1
                lis[i] = lis[j] + 1
    
    return max(lis)
\end{lstlisting}
% inc(A, i, max)
%     if i == 1
%         return 1
    
%     local_mx = 1
%     for j from 1 to (n - 1)
%         val = inc(A, j, max)
%         if A[j-1] < A[i-1] && val > local_mx - 1
%             local_mx = val + 1
    
%     if max < local_mx
%         max = local_mx
    
%     return local_mx

    % b
    \item With $INC[i]$, $LIS$ become quite handy by invoking $INC[n]$ where $n$ is the 
          last element of the input array. For $A = [x_1, x_2, ..., x_m], 
          LIS = max_{i=1, 2, ..., m}INC[i] \ where \
          INC[i] = inc(A, i, 1)$. Similarly for $DIS$, with $DEC[i], DIS = max_{i=1,..., m}DEC[i]$. 

    % c
    \item $INC[i] = 1 + max(INC[j]) \geq INC[j] + 1$ , where $1<j<i, A[i] > A[j]$. So it is
          impossible to have $INC[i]=INC[j]$. Same for $DEC[i]=DEC[j]$. 
    
          Or we can prove by assuming we can get $INC[i]=INC[j] \ where \ i > j$. 
          If $A[i] \geq A[j]$, $INC[i] = INC[j] + 1$ since $A[i]$ would be added to the end of 
          the subsequence, and we get the subsequence ended at $A[i]$ with length $INC[j] + 1$. If 
          $A[i] < A[j]$, the subsequence end in $A[i]$ would be totally different than $A[j]$. 

    % d
    \item We can prove by pigeonhole principle. 
          Assume $LIS \leq a$ and $DIS \leq b$. Label each element $A[i]$ with a pair $(INC[i], DEC[i])$. 
          Picking 2 numbers $0 < j < i < m + 1$,  if $\ A[j] < A[i] \ then \ INC[i] \geq INC[j] + 1$.
          Otherwise, if $A[j] > A[i], DEC[j] \geq DEC[i] + 1$. 
          
          But there are only $ab$ possible numbers and $INC[i] \leq a$ and $DEC[i] \leq b$. 
          $ab < ab + 1$ or $ab < ab + 1$, 
          so there must exist an $i$ in either $INC[i]$ or $DEC[i]$ would be out of the bound of index. 
          $A[i]$ is in $INC[i] \ and \ INC[i] \geq (a+1)$ if $INC[i]$ steps out of the bound, or 
          $B[i]$ is in $DEC[i] \ and \ DEC[i] \geq (b+1)$ if $DEC[i]$ steps out of the bound. 
          Thus, either $LIS > a \ or \ DIS > b$. 
          
          % $INC[i] \times DEC[i] \leq ab$. When 
    
          I read the Seidenberg, A. (1959) paper, "A simple proof of a theorem of Erdős and Szekeres". 
    
    % Seidenberg, A. (1959), "A simple proof of a theorem of Erdős and Szekeres", Journal of the London Mathematical Society 34: 352, doi:10.1112/jlms/s1-34.3.352
\end{enumerate}

    
\end{homeworkProblem}

%----------------------------------------------------------------------------------------
%	PROBLEM 4
%---------------------------------------------------------------------------------------

\begin{homeworkProblem}
By MATRIX-CHAIN-ORDER(p) where $p = [5, 10, 3, 12, 5, 50, 6]$. 
\[
 m[i, j] =
  \begin{cases}
   0 & \text{if } \text{if } i = j\\
   min_{i\leq k < j}(min(i,k) + m(k+1, j) +p_{i-1}p_kp_j)       & \text{if } i < j
  \end{cases}
\]

$m[i, i] = 0$ for all i. Then we can compute $m[i, i + 1]$ for i = 1, ..., $n – 1$.
The resulting table would be:\\

The m-table:\\
m[1, 2] = 150, m[2, 3] = 360, m[3, 4] = 180, m[4, 5] = 3000, m[5, 6] = 1500 \\
m[1, 3] = 330, m[2, 4] = 330, m[3, 5] = 930, m[4, 6] = 1860 \\
m[1, 4] = 405, m[2, 5] = 2430, m[3, 6] = 1770 \\
m[1, 5] = 1655, m[2, 6] = 1950 \\
m[1, 6] = 2010 \\

The s-table: \\
s[1, 2] = 1, s[2, 3] = 2, s[3, 4] = 3, s[4, 5] = 4, s[5, 6] = 5 \\
s[1, 3] = 2, s[2, 4] = 2, s[3, 5] = 4, s[4, 6] = 4 \\
s[1, 4] = 2, s[2, 5] = 2, s[3, 6 = 4 \\
s[1, 5] = 4, s[2, 6] = 2 \\
s[1, 6] = 2 \\

We have (A 1 A 2 )((A 3 A 4 )(A 5 A 6 )).

\end{homeworkProblem}

%----------------------------------------------------------------------------------------
%	PROBLEM 5
%----------------------------------------------------------------------------------------
\begin{homeworkProblem}
\begin{enumerate}[(a)]
    \item $\lg(4^n / \sqrt n) = \lg(2^2n) - \lg(\sqrt n) = 2n - 1/2\lg(n) = O(n)$
    
    \item By taking the log value of both \\
          $313340 \lg5 = 219015.261158648$ \\
          $271251 \lg7 = 229233.688451907$ \\
          $7^{271251}$ is bigger. 
          
    \item $n^2\lg(n^2) = 2n^2\lg(n)$ \\
          $\lg^2(n^3) = (\lg(n^3))^2 = (3\lg(n))^2 = 9\lg^2(n)$
          
    \item $$ \ln(e^{\frac{-x^2}{2}}) = \ln(\frac{1}{n})$$
          $$ \frac{-x^2}{2} = -\ln(n)$$
          $$ x^2 = 2\ln(n)$$
          $$ x = \sqrt{2\ln(n)} \ or \ -\sqrt{2\ln(n)}$$
          
    \item $\log_n2^n = n \log_n 2 = \frac{n}{\log_2 n}$
          $= \log_nn^2 = 2$
          
    \item Since $\lg n = \log_2 n$, $\frac{\log_2 n}{log_3 n}  
          = \frac{\log_2 n}{\frac{\log_2 n}{\log_2 3}} 
          = \log_2 3 > \log_2 2 = 1$. So $\lg n > \log_3 n$. 
    
    \item $i \times 2^x \geq n$, $x = \lceil log_2 \frac{n}{i} \rceil$. 
    
    \item $\lg[n^ne^{-n}\sqrt{2\pi n}] 
          = n\lg(n) - n \lg e + \frac{1}{2}lg(2\pi n) 
          = (n+\frac{1}{2})\lg n - n\lg e + \frac{1}{2}(1+\lg \pi)
          = O(n\lg(n)) $. It's the Sterling theorem. 
          We can get a polynomial. 
\end{enumerate}
\end{homeworkProblem}


% \begin{lstlisting}[frame=single]
% \end{lstlisting}

% \begin{enumerate}[a.]
%     \item 
        %   
        %   
% \end{enumerate}
\end{document}