\documentclass{article}
\usepackage{amsmath}
\usepackage{enumerate}
\usepackage{fancyhdr} % Required for custom headers
\usepackage{lastpage} % Required to determine the last page for the footer
\usepackage{extramarks} % Required for headers and footers
\usepackage[usenames,dvipsnames]{color} % Required for custom colors
\usepackage{graphicx} % Required to insert images
\usepackage[tight,footnotesize]{subfigure} % Required for subfig
\usepackage{caption} % Required for subfig
\usepackage{hyperref} % Required for url
\usepackage{listings} % Required for insertion of code
\usepackage{courier} % Required for the courier font
\usepackage{lipsum} % Used for inserting dummy 'Lorem ipsum' text into the template
\topmargin=-0.45in
\evensidemargin=0in
\oddsidemargin=0in
\textwidth=6.5in
\textheight=9.0in
\headsep=0.25in
\linespread{1.1} % Line spacing
\pagestyle{fancy}
\lhead{\hmwkAuthorName} % Top left header
% \chead{\hmwkClass\ (\hmwkClassInstructor\ \hmwkClassTime): \hmwkTitle} % Top center head
\chead{\hmwkClass\ : \hmwkTitle} % Top center head
\rhead{\firstxmark} % Top right header
\lfoot{\lastxmark} % Bottom left footer
\cfoot{} % Bottom center footer
\rfoot{Page\ \thepage\ of\ \protect\pageref{LastPage}} % Bottom right footer
\renewcommand\headrulewidth{0.4pt} % Size of the header rule
\renewcommand\footrulewidth{0.4pt} % Size of the footer rule
\setlength\parindent{0pt} % Removes all indentation from paragraphs

% Define floor and ceiling
\def\lc{\left\lceil}   
\def\rc{\right\rceil}
\def\lf{\left\lfloor}   
\def\rf{\right\rfloor}

% Set your language 
%\lstset{language=Java}
\definecolor{codegreen}{rgb}{0,0.6,0}
\definecolor{codegray}{rgb}{0.5,0.5,0.5}
\definecolor{codepurple}{rgb}{0.58,0,0.82}
\definecolor{backcolour}{rgb}{0.95,0.95,0.92}
 
\lstdefinestyle{mystyle}{
    backgroundcolor=\color{backcolour},   
    commentstyle=\color{codegreen},
    keywordstyle=\color{magenta},
    numberstyle=\tiny\color{codegray},
    stringstyle=\color{codepurple},
    basicstyle=\footnotesize,
    breakatwhitespace=false,         
    breaklines=true,                 
    captionpos=b,                    
    keepspaces=true,                 
    numbers=left,                    
    numbersep=8pt,                  
    showspaces=false,                
    showstringspaces=false,
    showtabs=false,                  
    tabsize=2
}
\lstset{style=mystyle}

% Header and footer for when a page split occurs within a problem environment
\newcommand{\enterProblemHeader}[1]{
\nobreak\extramarks{#1}{#1 continued on next page\ldots}\nobreak
\nobreak\extramarks{#1 (continued)}{#1 continued on next page\ldots}\nobreak
}

% Header and footer for when a page split occurs between problem environments
\newcommand{\exitProblemHeader}[1]{
\nobreak\extramarks{#1 (continued)}{#1 continued on next page\ldots}\nobreak
\nobreak\extramarks{#1}{}\nobreak
}

\setcounter{secnumdepth}{0} % Removes default section numbers
\newcounter{homeworkProblemCounter} % Creates a counter to keep track of the number of problems

\newcommand{\homeworkProblemName}{}
\newenvironment{homeworkProblem}[1][Problem \arabic{homeworkProblemCounter}]{ % Makes a new environment called homeworkProblem which takes 1 argument (custom name) but the default is "Problem #"
\stepcounter{homeworkProblemCounter} % Increase counter for number of problems
\renewcommand{\homeworkProblemName}{#1} % Assign \homeworkProblemName the name of the problem
\section{\homeworkProblemName} % Make a section in the document with the custom problem count
\enterProblemHeader{\homeworkProblemName} % Header and footer within the environment
}{
\exitProblemHeader{\homeworkProblemName} % Header and footer after the environment
}

\newcommand{\problemAnswer}[1]{ % Defines the problem answer command with the content as the only argument
\noindent\framebox[\columnwidth][c]{\begin{minipage}{0.98\columnwidth}#1\end{minipage}} % Makes the box around the problem answer and puts the content inside
}

\newcommand{\homeworkSectionName}{}
\newenvironment{homeworkSection}[1]{ % New environment for sections within homework problems, takes 1 argument - the name of the section
\renewcommand{\homeworkSectionName}{#1} % Assign \homeworkSectionName to the name of the section from the environment argument
\subsection{\homeworkSectionName} % Make a subsection with the custom name of the subsection
\enterProblemHeader{\homeworkProblemName\ [\homeworkSectionName]} % Header and footer within the environment
}{
\enterProblemHeader{\homeworkProblemName} % Header and footer after the environment
}

\newlength{\tabcont}

\newcommand{\tab}[1]{%
\settowidth{\tabcont}{#1}%
\ifthenelse{\lengthtest{\tabcont < .25\linewidth}}%
{\makebox[.25\linewidth][l]{#1}\ignorespaces}%
{\makebox[.5\linewidth][l]{\color{red} #1}\ignorespaces}%
}%
%----------------------------------------------------------------------------------------
%	NAME AND CLASS SECTION
%----------------------------------------------------------------------------------------

\newcommand{\hmwkTitle}{Homework\ \#4} % Assignment title
\newcommand{\hmwkDueDate}{Monday,\ January\ 1,\ 2012} % Due date
\newcommand{\hmwkClass}{Fundamental Algorithms} % Course/class
\newcommand{\hmwkClassTime}{} % Class/lecture time
\newcommand{\hmwkClassInstructor}{Prof. Joel Spencer} % Teacher/lecturer
\newcommand{\hmwkAuthorName}{Songxiao Zhang, N10224459, {\tt 72}} % Your name

%----------------------------------------------------------------------------------------
%	TITLE PAGE
%----------------------------------------------------------------------------------------

\title{
\textmd{\textbf{\hmwkClass:\ \hmwkTitle}}\\
}
\author{\textbf{\hmwkAuthorName}}

\begin{document}

\maketitle

%----------------------------------------------------------------------------------------
%	PROBLEM 1
%----------------------------------------------------------------------------------------
\begin{homeworkProblem}
$T(n) = 9T(n/3) + n^2$  \\
$T(1) = 1$  \\
$T(3) = 9T(3/3) + 3^2 = 9*T(1) + 9  = 9*1 + 9  = 18$   \\
$T(9) = 9T(9/3) + 9^2 = 9*T(3) + 81 = 9*(9*1 + 9) + 81 = 243$ \quad  \\
$T(27) = 9T(27/3) + 27^2 = 9*T(9) + 729 = 9*(9*(9*1 + 9)+81) + 729 = 2916$   \\
$T(81) = 9T(81/3) + 81^2 = 9*T(27) + 6561 = 9*2916 + 6561 = 32805$   \\
$T(243)= 9T(243/3) + 243^2 = 9*T(81) + 59049 = 9*32805 + 59049 = 354294$  \\
A general guess for $T(3^i)$ where $i=log_3 n$ would be
$$ T(3^i) = \sum\limits_{j=0}^{i-1} (9^{i-j} \times 3^j + 3^{2i} )
= \sum\limits_{j=0}^{i-1} (3^{2i-j} + 3^{2i} ) $$

$$ T(n) = \sum\limits_{j=0}^{log_3 n - 1} (9^{log_3 n - j} \times 3^j  + n^2)
    = \sum\limits_{j=0}^{log_3 n - 1} (3^{2log_3 n - j}  + n^2)$$
    
By approaching from left side
$ T(3^i) > \sum\limits_{j=0}^{i-1} (3^{2i} ) 
        % = \frac{3^{2(i-1)} - 1}{3^2 - 1}
        % = \frac{3^{2i} - 1}{8} $, from right side 
          = i3^{2i} $, from right side 
$ T(3^i) < \sum\limits_{j=0}^{i-1} (2 \times 3^{2i} ) 
        %  = \frac{3^{2(i-1)-1}}{3^2 - 1}
        %  = \frac{3^{2i} - 1}{4} $. \\
         = 2i3^{2i} $. \\
So $T(3^i) = \Theta(i3^{2i})$. $T(n) = \Theta(n^2 lg\ n)$. \\\

By Master Theorem, $T(n) = 9T(n/3) + n^2$, $a=9, b=3$, \\
$f(n) = \Theta(n^{log_b a}) = \Theta(n^2)$, then $T(n) = \Theta(n^{log_b a}lg\ n) = \Theta(n^2 lg\ n)$
% n = 3^2, 9*(n/3)**2 + 9**2*(n/3^2) + n**2
% n = 3^3, 9*(n/3)**2 + 9**2*(n/3)   + 9**3*(n/3**3) + n**2
% n = 3^4, 9*(n/3)^2+9^2*(n/3^2)+9^3*(n/3^3)
\end{homeworkProblem}

%----------------------------------------------------------------------------------------
%	PROBLEM 2
%----------------------------------------------------------------------------------------
\begin{homeworkProblem}
\begin{enumerate}[a.]
    \item By Master Theorem, $T(n) = 6T(n/2) + n\sqrt{n}$, $a=6, b=2$, \\
          $f(n) = n^{1.5} = O(n^{log_b a - \epsilon}) = O(n^{log_2 6 - (log_2 6 - 1.5)})$, 
          then $T(n) = \Theta(n^{log_b a}) = \Theta(n^{log_2 6})$
    \item By Master Theorem, $T(n) = 4T(n/2) + n\sqrt{n}$, $a=4, b=2$, \\
          $f(n) = n^{5} = O(n^{log_b a + \epsilon}) = O(n^{2+3})$, 
          $4(n/2)^5 = n^5/2^3 \leq n^5/2^3$ and $1/2^3 < 1$ since $af(n/b) \leq cf(n)$ 
          then $T(n) = \Theta(f(n)) = \Theta(n^5)$
    \item By Master Theorem, $T(n) = 4T(n/2) + 7n^2 + 2n + 1$, $a=4, b=2$, \\
          $f(n) = \Theta(n^{2}) = O(n^{log_b a}) = O(n^{2})$, 
          then $T(n) = \Theta(n^{log_b a}lg n) = \Theta(n^{2}lgn)$
\end{enumerate}
\end{homeworkProblem}

%----------------------------------------------------------------------------------------
%	PROBLEM 3
%----------------------------------------------------------------------------------------
\begin{homeworkProblem}
Instead of divide the number into 2 like Karatsuba algorithm did ($X = x_1 \times x + x_0$), 
Toom-3 divide the number into 3 ($X = x_2 \times x^2 + x_1 \times x + x_0$) so in each digit it takes
$T(\frac{n}{3})$, while Karatsuba took $T(\frac{n}{2})$. 

Karatsuba has 3 recursions $3T(\frac{n}{2})$ and 2 additions ($O(n)$). Toom-3 has 5 recursions $5T(\frac{n}{3})$ and 30 additions and subtractions ($O(n)$). 

Thus, by Master Theorem, the running time of Toom-3 $$T(n) = 5T(\frac{n}{3}) + \Theta(n) = \Theta(n^{log_3 5}). $$ Toom-3 algorithm would be faster than Karatsuba algorithm when $n$ gets large. 
\end{homeworkProblem}

%----------------------------------------------------------------------------------------
%	PROBLEM 4
%----------------------------------------------------------------------------------------
\begin{homeworkProblem}
\begin{enumerate}[a.]
    \item $T(n) = n^2+(n+1)^2+\ldots + (2n)^2 
          = \sum\limits_{i=0}^{2n} i^2 - \sum\limits_{i=0}^{n-1} i^2$, since
          $\sum\limits_{i=0}^{n} i^2 = \frac{n(n+1)(2n+1)}{6}$. 
          $$T(n) = \frac{2n(2n+1)(4n+1)}{6} - \frac{(n-1)(n-1+1)(2n-2+1)}{6} = \frac{14n^3+15n^2-n}{6}$$
          $g(n) = \Theta(n^3), c_1 = 2, c_2 = 3 $. 
    \item $T(n) = \lg^2(1)+\lg^2(2)+\ldots + \lg^2(n) = \sum\limits_{i=0}^{n} lg^2 i $. 
          Approach from right $T(n) \leq nlg^2 n$, and approach from left by deleting first $n/2$ itmes
          $T(n) > (n/2) * lg^2(n/2)$ but $\frac{1}{2}$ can never get, we can take $\frac{1}{4}$ as a smaller number that we can reach. So we can get $g(n) = \Theta(nlg^2n), c_1 = 1/4, c_2 = 1 $. 
    \item $T(n)=1^3+\ldots+n^3=\sum\limits_{i=1}^{n} \frac{n^2 \times (n+1)^2}{4} = \frac{n^4+2n^3+n^2}{4}$
          so $g(n) = \Theta(n^4), c_1 = 1/4, c_2 = 1 $
\end{enumerate}
\end{homeworkProblem}

%----------------------------------------------------------------------------------------
%	PROBLEM 5
%----------------------------------------------------------------------------------------
\begin{homeworkProblem}
\begin{lstlisting}[frame=single]
// Assume Integer.parseInt(A.toString()) - Integer.parseInt(B.toString()) > 0 
for (int i = N; i >= 0; i--) {
    if (A[i] >= B[i])
        C[i] = A[i] - B[i];
    else {
        C[i] = A[i] + 10 - B[i];
        A[i+1]--;
    }
}
\end{lstlisting}
It took $N-1$ times so $T(n) = \Theta(n)$. 
\end{homeworkProblem}

% \begin{lstlisting}[frame=single]
% \end{lstlisting}

% \begin{enumerate}[a.]
%     \item 
% \end{enumerate}
\end{document}